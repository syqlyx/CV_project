\documentclass[10pt,a4paper]{article}
\usepackage{geometry}
\usepackage[utf8]{inputenc}
\usepackage{CJKutf8}
\usepackage{graphicx}
\usepackage{epsfig}
\usepackage{mathptmx}
\usepackage{times}
\usepackage{amsmath}
\usepackage{amssymb}
\usepackage{cite}
\usepackage{subfigure}
\usepackage{verbatim}
\usepackage{multirow}
\usepackage{bm}
\usepackage{float}
\usepackage{etoolbox,xstring,mfirstuc,textcase}
\usepackage{indentfirst}
\usepackage{enumerate}
\author{沈艳晴}
\title{Assignment1--相机参数标定}
\geometry{left=3.18cm, right=3.18cm, top=2.54cm, bottom=2.54cm}

\begin{document}
\begin{CJK*}{UTF8}{gbsn}
\CJKindent%中文缩进专用
\maketitle

%分别有编号,对齐
\begin{align}
a&=1 \\
b&=2
\end{align}
%无编号,突出显示,对齐
$$
\begin{aligned}
a&=1 \\
b&=2
\end{aligned}
$$
%有一个编号,对齐
\begin{equation}
\begin{aligned}
a&=1 \\
b&=2
\end{aligned}
\end{equation}
%不突出显示,行内上下对齐
what i mean is that $
\begin{aligned}
a&=1 \\
b&=2
\end{aligned}
$, therefore

$$
\begin{array}{cc}
1 & 2\\2&3 \\2&6
\end{array}
$$

$$
\left[
\begin{array}{c}
1 \\ 2 \\ 3
\end{array}\right]
$$

$$
\begin{bmatrix}
1 \\ 2 \\ 3
\end{bmatrix}
$$

\[
\begin{matrix}
1 & 2\\2&3 \\2&6
\end{matrix}
\]
%公式内不允许空行


\section{aa} \label{aa}


%\pageref{•}%返回页码

%\ref{•}%返回编号,但对于浮动体可能会返回所在节[图、表]

section~ \ref{aa}


%%%%{
交叉引用(cross-referencing)的三个命令:
/label{marker}: 对可能引用的对象一个标记,可以认为是名字
/ref{marker}: 引用之前定义为marker的对象,输入该对象被赋予的编号
(ref) == eqref
/pageref{marker}:打印对象所在的页码

a)Latex 自动计算编号 b) 通过一次编译,生成标号,二次编译将/ref{marker}换成数字(两次编译)(只有一次编译,可能会引用未更新的编号) c) 如果引用的标记不存在,latex编译会通过,但是提示一个警告,同时将对应的/ref{marker}用??换下

原则上/label位置很灵活,但是建议直接紧跟所要指代的对象。特别的,如果一个标签在浮动环境中被声明的话,只有当它紧跟在caption{}之后,/ref{marker} 才会返回对应的图片或者表格的编号。否则,即声明写在浮动环境]外,/ref{marker}会直接返回所在section的编号。
原因在于latex只有遇到/caption之后才给对象添加计数器。因此,/label在前的话,对应的/ref会被迫引用上级元素的编号。




















\end{CJK*}
\end{document}